%%%%%%
%%
%%  Don't reorder the reviewer points; that'll mess up the automatic referencing!
%%
%%%%%

\begin{minipage}[b]{2.5in}
  Cover Letter \\
  {\it Evolution}
\end{minipage}
\hfill
\begin{minipage}[b]{2.5in}
    Joshua Schiffman \\
    \emph{and} Peter Ralph \\
  \today
\end{minipage}
 
\vskip 2em
 
\noindent
{\bf To the Editor(s) -- }
 
\vskip 1em

Explain this was split from the other paper.

\noindent \hspace{4em}
\begin{minipage}{3in}
\noindent
{\bf Sincerely,}

\vskip 2em

{\bf 
Joshua Schiffman and
Peter Ralph
}\\
\end{minipage}

\vskip 4em

\pagebreak

%%%%%%%%%%%%%%
\reviewersection{AE}

\begin{quote}
Your manuscript is quite long already, and the reviewers' suggestions of
modifications and clarification, which need to be implemented, will make the
manuscript even longer. I however share the reviewers' opinion that the
manuscript's two parts are only loosely related. My suggestion is therefore to
publish them separately. Although connecting the systems biology and the
quantitative genetics parts is indeed an exciting endeavor, this is not really
achieved in the current version of the manuscript, and it would be more
profitable to first better describe each part separately. Regarding the
connexion between the two parts, please also pay particular attention to R2's
first specific comment about dimensionality.
\end{quote}

\begin{point}{p11, 2nd paragraph}
    Try to better relate these different studies to yours
    (e.g., what you add, what they do and you do not). In addition to R1's
    suggestions, consider \citet{weinreich2013fishers} and \citet{blanquart2016epistasis}.
% Weinreich, D. M. and Knies, J. L. (2013), FISHER'S GEOMETRIC MODEL OF ADAPTATION MEETS THE FUNCTIONAL SYNTHESIS: DATA ON PAIRWISE EPISTASIS FOR FITNESS YIELDS INSIGHTS INTO THE SHAPE AND SIZE OF PHENOTYPE SPACE. Evolution, 67: 2957-2972. doi:10.1111/evo.12156
% François Blanquart and Thomas Bataillon Epistasis and the Structure of Fitness Landscapes: Are Experimental Fitness Landscapes Compatible with Fisher’s Geometric Model?  GENETICS Early online April 6, 2016
\end{point}

\reply{
}


\begin{point}{Hybridization:}
    What is $\mathcal{X}$?
\end{point}

\reply{
    A holdover from previous notation; it is $\allS$ (fixed). \revref
}

\begin{point}{p15}
    ``The importance of including neutral directions in these models, which
    is not usually done'' -- Do not lines of isofitness correspond to neutral
    directions, and aren't these neutral directions already included in those models?
\end{point}

\reply{
    Good point! We meant neutral directions \emph{within the optimal set},
    and have reworded this.
}


%%%%%%%%%%%%%%
\reviewersection{1}

\begin{quote}
    \textbf{(II)} The popgen part. This part models the ``system drift'', which is nothing else
but neutral drift of a population along a high-fitness ridge in an epistatic
landscape (conceptual figure 4), right? If two populations in allopatry drift
in different directions, this can lead to hybrid incompatibilities, which are
uncovered upon secondary contact. Fitness is modeled by the weighted distance
from the optimal impulse response function (eq 5). This is a natural
assumption.
\end{quote}

Yes, exactly. This is a natural model,
so we were surprised to not find an analytic treatment of it
in the literature.

\begin{point}{}
    For system drift, you assume that ``Selection will tend to restrain this
motion, but movement along the optimal set $\allS$ is unconstrained, and so we expect
the population mean to drift along the optimal set like a particle diffusing.''
I see two major problems with this view. Both lead to slower divergence and
therefore run against your conclusions.
    \textit{(further points to follow)}
\end{point}

\reply{
}

\begin{point}{(drift along a ridge)}
    In the presence of epistasis, evolution on a neutral network (a high-fitness
    ridge) is \emph{not} due to drift alone, but also affected by (weak second order)
    selection in favor of mutational robustness / genetic canalization. In contrast
    to what you write, diffusion on the set of network coefficients corresponding
    to the optimal phenotype is not unbiased -- even if the optimal phenotypes do
    indeed all have the same fitness. Instead, selection drives the population to
    ``thicker'' parts of the network where the mean fitness of the population
    (including a cloud of mutants) is higher that on a narrow ridge. This is the
    basis of the evolution of robustness/canalization. It is possible to account
    for this effect, see Hermisson et al 2003. Amer. Nat. 161, 708-734;
    Alvarez-Castro et al, TPB 75 (2009) 109-122, or also Rice, 1998, Evolution 52,
    647-656; van Nimwegen, etal 1999, PNAS 96:9716–9720. Note that epistasis is
    necessary for the neutral evolution of incompatibilities, which is what you are
    aiming for.
\end{point}

\reply{
}

\begin{point}{(drift along a ridge)}
    A second problem results from the fact that evolution on a high-fitness ridge
    often requires coordinated changes at many loci. Take the oscillator system
    that you use as an example: simultaneous changes at two genes are required to
    maintain the phenotype. This leads to a phenomenon called ``adaptive inertia''
    (see Baatz and Wagner, TPB 51, 49-66 and Alvarez-Castro etal . TPB 75, 109-122),
    which effectively slows down the movement along the ridge considerably. This
    problem applies, in particular, in a ``house-of-cards'' mutation regime when
    where rarely two mutations occur together on the same haplotype. In small
    populations, it typically requires that populations drift through shallow
    fitness valleys. While this is possible, it slows down the process.  
\end{point}

\reply{
}

\begin{point}{(drift along a ridge)}
    The relevance of both effects could be studied by simulations in a simple
example (eg the oscillator that is used as an illustration in the ms anyway).
\end{point}

\reply{
}

\begin{point}{}
    If speciation due to accumulation of incompatibilities does not occur in
allopatry, but under (even weak) gene flow, some degree of positive selection
is always needed (Bank et al, Genetics 191, 845–863 for a proof). This also
means that even weak gene flow will counteract the process described in the ms.
\end{point}

\reply{
}

\begin{point}{}
    Population isolates and genetic load: Isn't this exactly ``founder effect
speciation''?
\end{point}

\reply{
}


%%%%%%%%%%%%%%
\reviewersection{2}

\begin{quote}
In the second part of the paper, motivated by the previous results, the authors explore a general
quantitative genetics model in which populations can drift stochastically near a set of equivalent
and optimal systems. Since the optimal set (or evolutionary ridge) is not closed under averaging or
recombination, two isolated populations can drift apart and accumulate enough genetic differences
so that they do not produce any viable offspring. Using some heuristics, several expressions are
derived to quantify the accumulation of genetic incompatibilities.

Overall, I think the paper is well written. The framework developed in the first part of the paper
is very interesting and that the analogy with system theory is quite enlightening. I am somehow
less convinced by the second part. First, I find the arguments a bit sketchy (see below) and not so
easy to follow. Secondly, it is not entirely clear to me what is the main contribution of this part
compared to previous works. It seems to me that the main result is somehow contains in the fact
that the variance (or ``segregation variance'') of an $F_2$ population is given by
$$\sigma^2_S + 4 \omega \sigma^2_N T / N_e$$
which was already derived Slatkin and Lande according to the authors (except for the explicit
expression of $\omega$, but again I am a little bit confused by the arguments derived in the appendix).

In summary, I think this paper could be a nice contribution to Evolution. However, I am also
convinced that the second part of the paper would require more work, or more explicit reference
to previous works (equation, section etc.) if the authors do not want to re-derive already existing
formula.
\end{quote}


\begin{point}{p8:}
    ``A Taylor expansion of $D(h_\epsilon)$ .. ''. It is not clear to me at all. Could you provide some
extra explanation (e.g., in the appendix)?
\end{point}

\reply{
}

\begin{point}{p11. Paragraph system drift.}
    ``move a random distance $\sigma$''. What is $\sigma$? I think it should
be $\sigma_N$ (the std deviation in the direction of the evolutionary ridge) to be consistent with
the assumption that the population drifts along the optimal set. I believe that this is what
is assumed thereafter. Also, the sentence ``It therefore seems .... as cloud of points of width
$\sigma$'' is not very accurate, since the covariance matrix is not the identity.
\end{point}

\reply{
}

\begin{point}{p 11.}
    Approximating the optimal set $\mathcal{N}$ by a quadratic surface should only be accurate if
we look at the genetic divergence at small time scales. This should be at least mentioned.
\end{point}

\reply{
}

\begin{point}{end of p11.}
    $1/( \frac{d}{du} D(x + uz))$ should be evaluated at $u = 0$.
\end{point}

\reply{
}

\begin{point}{p12. Third paragraph}
    $\sqrt{4 \omega T/N_e} \sim \gamma/\sigma_N$.
    I guess the underlying assumption here is that
    $\sigma_S \ll \sqrt{4 \omega \sigma^2_N T/N_e}$?
\end{point}

\reply{
}

\begin{point}{p12. before eq. 7.}
    $\mu = c_\mu \gamma T /N_e$. Why is $\mu$ proportional to $\gamma$?
\end{point}

\reply{
}

\begin{point}{Fig 7.}
    I have one important issue with this figure (and the assumptions of the underly-
ing quantitative genetic model). If one wants to be consistent with the assumption that
parental populations drift along the evolutionary ridge, I think one would need to assume
that selection is strong enough to constraint the mean of the population on the surface. This
would presumably require that $\sigma S /\gamma \ll 1$. First, I think this assumption (or something
alternative to that) should be made explicit in the text. Secondly, it seems to me that this
assumption is not satisfied for panel A and C: under the range of parameters proposed by
the authors, the parental populations could easily drift away from the optimal set, and in
particular, the heuristics derived in the main text would not be satisfied.
\end{point}

\reply{
}

\begin{point}{}
    Finally, it would be worth mentioning several old and recent works relating genetic drift
to speciation : Yamagushi and Iwasa, several articles by Gavrilets et al. (I think several
citations are missing here), and Mirò Pina and Schertzer.
\end{point}

\reply{
}




