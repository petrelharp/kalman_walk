\documentclass[11 pt]{article}
\usepackage{amsmath, amssymb, color, xcolor}
\usepackage{graphicx, wrapfig, float, caption, dsfont, bbm}
\usepackage{fullpage}
\usepackage[backref=page, hidelinks, colorlinks=true, citecolor=blue!60!black!100]{hyperref}
\usepackage{tikz}
\usetikzlibrary{arrows.meta, shapes}
\usepackage{caption, subcaption}
\usepackage{natbib} % gives us \citet: Author (year) and \citep: (Author; year)
\usepackage{authblk}

\newcommand{\plr}[1]{{\color{blue}\it #1}}
\newcommand{\jss}[1]{{\color{olive}\it #1}}
% \newcommand{\ddt}{\frac{d}{dt}}
\newcommand{\ddt}{\dot}
\newcommand{\ro}{{ro}}
\newcommand{\nro}{{\bar{r}o}}
\newcommand{\rno}{{r\bar{o}}}
\newcommand{\nrno}{{\bar{r}\bar{o}}}
\newcommand{\reachable}{\mathcal{R}}
\newcommand{\unobservable}{\bar{\mathcal{O}}}
\newcommand{\R}{\mathbb{R}}
\newcommand{\E}{\mathbb{E}}
\renewcommand{\P}{\mathbb{P}}
\newcommand{\pda}{\frac{\partial}{\partial A_{ij}}}
\newcommand{\ind}{\mathds{1}}

\newcommand{\A}{\mathcal{A}}
\newcommand{\diag}{\text{diag}}
\newcommand{\1}{\mathbbm{1}}

\DeclareMathOperator{\spn}{span}

\newtheorem{theorem}{Theorem}
\newtheorem{lemma}{Lemma}
\newtheorem{definition}{Definition}
\newtheorem{example}{Example}

\begin{document}

  \begin{itemize}
    \item Do gene networks drift into uneccesarily complex configurations? If the optimal dynamics of a system can be realized by $g$ genes and/or started in a state with only $g$ genes, during evolution, will the system be composed of $> g$ genes. How many more genes than necessary? 
    \item Are there forces constraining gene network size other than a fitness cost associated with the expense of uncessary energy? For instance, will genes in a non-minimal gene network contribute less overall to the system's output, diminising the cost of their removal (on average)? 
    \item as the population explores neutral network space, how often will the population be on or near a point on the manifold where a gene becomes unccessary? Does this probability go up significantly as unneccsary dimensionality goes up?
    \item Does the unncessary complexity of some gene networks confer evolvability advantages? 
  \end{itemize}

Let $\Omega_n$ be the set of all allowable networks: this is some bounded subset of $\R^{n \times n}$,
and let
  \begin{align*}
    \mathcal{F}_{n} = \left\{ A : C_{n}(zI-A)^{-1} B_{n} = H(z) \right\} \subseteq \Omega_n
  \end{align*}
be the $d_n$-dimensional manifold of equivalent systems, 
where $A$ is gene network (not necessarily minimal), and $H(z)$ is a description of the phenotype in the Laplace domain. In the non-minimal case, $C_{k}$ and $B_{k}$ are augmented with zeroes.
  \begin{align*}
    C_{n} &= \left[ \begin{array}{cc} C_{k}  \\ \hline 0 \end{array} \right], C_{i} \neq 0 \\
      B_{n} &= \left[ \begin{array}{ccc} B_{k} & \vert & 0 \end{array} \right], B_{i} \neq 0
  \end{align*}
Let, 
  \begin{align*}
        P_{i} : \mathcal{F}_{n} \rightarrow \Omega_{n-1}
  \end{align*}
be the projection from an $n \times n$ network to an $(n-1) \times (n-1)$ network
obtained by removing the $i^\text{th}$ row and column. 
Biologically, this is a gene deletion or removal from the system.

  Let $G_{n,i}$ be the set of networks with identical phenotypes following a deletion. 
  \begin{align*}
        G_{n,i} = \left\{ f \in \mathcal{F}_{n} : P_{i}f \in \mathcal{F}_{n-1} \right\} \\
        G_{n} = \cup_{i} G_{n,i} \\
        d(f) = \# \left\{ i : P_{i}f \in \mathcal{F}_{n-1} \right\} 
  \end{align*}
$d(f)$ is the number of different genes that can potentially be deleted to end up in a phenotypically identical space. 

Our basic question is: for a given $\epsilon$,
how much of $\mathcal{F}_n$ is within $\epsilon$ of $G_n$?
The exact answer will depend on the details, but maybe we see how this changes with $n$.

One way of getting at this is to look in the other direction.
For instance, we know immediately a set of networks in which a gene can be deleted with no effect:
it is those networks that just had a gene added.
Given a network $A$ we can add a new gene without changing its phenotype
by inserting an arbitrary row $x$ into the $i^\text{th}$ slot
and a corresponding column of zeros.
We denote this operation $U_i(x)$, and:
  \begin{align*}
        U_{i}(x) \mathcal{F}_{n-1} \rightarrow \mathcal{F}_{n} \\
        \left\{ U_{i}(x) f : x \in \mathbb{R}^{n} \; f \in \mathcal{F}_n \right\} \subseteq G_{n,i} 
  \end{align*}
(Note that gene duplication would do something more specific
if we assume the new gene copy does the same thing as the old one.)
This implies that $d_n \ge d_{n-1} + n$.


Now we want to answer one of two questions:

How much of a $d_{n}$-dimensional manifold is within $\varepsilon$ of one of $(n-k)$ sub-manifolds, each of dimension $(d_{n-1}+n)$?
This scales like
\begin{align*}
(n-k)\varepsilon^{d_{n}-d_{n-1} - n}
\end{align*}

Alternatively, what is the hitting time of $U(x)$?


\end{document}
