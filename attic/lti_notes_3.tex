\documentclass[a4paper, 11 pt]{article}
\usepackage{amsmath, amssymb, color, setspace, graphicx, dsfont, pdfpages, float, wrapfig, indentfirst, epstopdf, inputenc, multicol}
\usepackage[left=0.5in, right=0.5in]{geometry}
\usepackage[font=small,labelfont=bf]{caption}
\begin{document}

\textbf{Zero State Definition:}

If a system is in the zero-state then the zero-input will yield a null response. 

\textbf{Zero-State Response Definition:}

The zero-state response of a system to $\delta(t-\xi)$ is, 
\begin{equation}
  \begin{split}
  \delta(t-\xi) = \int_{t_{0}}^{t} f(t, \xi^{'})\delta(\xi^{'}-\xi)d\xi^{'} \\
   = f(t, \xi) \text{ for } t \geq \xi \geq t_{0} \\
   = 0 \text{ for } \xi > t \geq t_{0}
 \end{split}
\end{equation}

Therefore the zero-state response of a system to input $u$ is, 
\begin{equation}
  A(u) = \int_{t_{0}}^{t}f(t,\xi)u(\xi)d\xi \ \ t_{0} \leq \xi \leq t
\end{equation}

A system is zero-state time invariant if and only if its impulse response $f(t,\xi)$ is of the form $f(t-\xi)$. 

\textbf{Proof:}

    Let $f(t,\xi) = w(\tau, t)$

    where $\tau \triangleq (t -\xi)$

    The zero-state response of a system to $\delta(t-\xi)$ is $f(t,\xi)$

    thus the zero-state response of the system to $\delta(t -(\xi + \lambda))$ where $\lambda$ is an arbitrary shift is $f(t, \xi + \lambda)$ or $w(\tau -\lambda,t)$

    this implies $w(\tau - \lambda, t) = w(\tau - \lambda, t - \lambda) \ \forall t \forall \tau \forall \lambda$

\textbf{Transfer Function Definition:}
\begin{equation}
  G(s) \triangleq \int_{-\infty}^{\infty} e^{-st}f(t)dt
\end{equation}

If $y$ is the zero-state response to $u$ then, 

\begin{equation}
  Y(s) = G(s)U(s)
\end{equation}

The Transfer Function gives incomplete information about the zero-input response and fully characterizes the zero-state response. 

The Transfer Function can also be viewed as the steady state response of a system to input $e^{st}$ divided by $e^{st}$:
$ G(s) \triangleq \frac{\text{steady state response of a system to }e^{st}}{e^{st}}$

Zero-State Equivalent Linear Dynamical Systems.

\begin{multicols}{2}
\begin{equation}
    \Sigma  
    \left \{ 
      \begin{array}{ll}
        \dot{x} = Ax + Bu \\
        y = Cx
      \end{array}
      \right.
\end{equation}

  \begin{equation}
  \bar{\Sigma}
    \left \{ 
      \begin{array}{ll}
        \dot{\bar{x}} = \bar{A}\bar{x} + \bar{B}u \\
        y = \bar{C}\bar{x}
      \end{array}
      \right. 
\end{equation}
\end{multicols}

\textbf{Definition:} two dynamical systems $\Sigma$ and $\bar{\Sigma}$ are algebraically equivalent if (where $\bar{x} = Px$),

  (a) $\bar{A} = PAP^{-1}$

  (b) $\bar{B} = PB$

  (c) $\bar{C} = CP^{-1}$

  Where $P$ is a nonsingular square matrix of rank $n$. If two systems are algebraically equivalent then they are also zero-state equivalent. Two systems that are zero-state equivalent are not necessarily algebraically equivalent and can be in different dimensions. 

  Definition: two dynamical systems are zero-state equivalent if they have the same transfer function $G(s)$. 

  The transfer function, $G(s) = C(sI-A)^{-1}B$, is the Laplace transform of the system's time domain dynamics.

  $G(s) = \bar{G}(s) \iff$

  $CA^{i}B = \bar{C}\bar{A}^{i}\bar{B}$ for all $i \geq 0$.

  \textbf{Proof:}

  $G(s) = C(sI-A)^{-1}B$ 

  $=Cs^{-1}(I-s^{-1}A)^{-1}B$

$=Cs^{-1} \left( \sum_{i=0}^{\infty}(s^{-1}A)^{i}\right)B$

$=\sum_{i=0}^{\infty} CA^{i}Bs^{i+1}$


$\sum_{i=0}^{\infty} CA^{i}Bs^{-(i+1)} = \sum_{i=0}^{\infty} \bar{C}\bar{A}^{i}\bar{B}s^{-(i+1)}$

$\iff CA^{i}B = \bar{C}\bar{A}^{i}\bar{B} \ \forall i$


Note: the controllability matrix $\mathcal{C}$ is defined as:
\begin{equation}
  \begin{bmatrix}
      B \ AB \ ... \ A^{n-1}B
  \end{bmatrix}
\end{equation}
And the observability matrix $\mathcal{O}$ is defined as:
\begin{equation}
    \begin{align}
  \begin{bmatrix}
      C \\ CA \\ \vdots \\ CA^{n-1}
  \end{bmatrix}
    \end{align}
\end{equation}
\end{document}
