\documentclass[11 pt]{article}
\usepackage{amsmath, amssymb, color, xcolor}
\usepackage{graphicx, wrapfig, float, caption, dsfont, bbm}
\usepackage{fullpage}
\usepackage[backref=page, hidelinks, colorlinks=true, citecolor=blue!60!black!100]{hyperref}
\usepackage{tikz}
\usetikzlibrary{arrows.meta, shapes}
\usepackage{caption, subcaption}
\usepackage{natbib} % gives us \citet: Author (year) and \citep: (Author; year)
\usepackage{authblk}

\newcommand{\plr}[1]{{\color{blue}\it #1}}
\newcommand{\jss}[1]{{\color{olive}\it #1}}
% \newcommand{\ddt}{\frac{d}{dt}}
\newcommand{\ddt}{\dot}
\newcommand{\ro}{{ro}}
\newcommand{\nro}{{\bar{r}o}}
\newcommand{\rno}{{r\bar{o}}}
\newcommand{\nrno}{{\bar{r}\bar{o}}}
\newcommand{\reachable}{\mathcal{R}}
\newcommand{\unobservable}{\bar{\mathcal{O}}}
\newcommand{\R}{\mathbb{R}}
\newcommand{\E}{\mathbb{E}}
\renewcommand{\P}{\mathbb{P}}
\newcommand{\pda}{\frac{\partial}{\partial A_{ij}}}
\newcommand{\ind}{\mathds{1}}

\newcommand{\A}{\mathcal{A}}
\newcommand{\diag}{\text{diag}}
\newcommand{\1}{\mathbbm{1}}

\DeclareMathOperator{\spn}{span}

\newtheorem{theorem}{Theorem}
\newtheorem{lemma}{Lemma}
\newtheorem{definition}{Definition}
\newtheorem{example}{Example}

\begin{document}

  \section{Draft}

  In order to survive -- that is to aquire enough energy to replicate -- an organism needs to execute at least one of a large number of potential survival strategies. The set of potential strategies includes which molecules to metabolize and which behaviors to exhibit in response to stimuli, among many others. To play at least one of these strategies an organism must organize its phenotype in a highly specific way. Presumably this is largely accomplished via the coordinated expression of evolutionarily pliable gene networks. We can say that a given environment determines a set of strategies, and an organism's fitness is determined by its phenotype's ability to execute one or more of such strategies effectively. An environment with two different available types of sugar can be viewed as (at minimum)  a two strategy environment. The successful organism can play one strategy or the other, or both strategies, either metabolizing one or both of the sugars. In this simplified world, adaptation includes the processes of improving existing, as well as finding novel strategies to play (although the energy required to forge novel strategies may well be too high if is not precipitated by environmental or other change rendering current strategies ineffective). It should also be noted that depending on the strategy, more than one type of phenotype may be equally suitable (\emph{e.g.} different metabolic networks processing the same metabolite may be equally energetically efficient).

  We aim to understand how populations evolve. To begin, we focus on the simplest case: the evolution of a large and perfectly adapted population in a constant one strategy, one phenotype environment for an indefinite number of generations. Under these conditions we expect a conservation of phenotype as no adaptation is possible and genetic drift will be very rare. Despite this conservation at the phenotypic level we can still expect significant evolutionary change at the genetic level, in most cases. In more complex scenarios (with varying phenotypes) we would observe conservation of strategy, or more broadly (with varying strategies), conservation of fitness. 

   Ordinarily, biologists distinguish only phenotype from genotype, however, here we will introduce (and justify the necessity of) an organism's \emph{kryptotype}. If one imagines an organism to be a \emph{black box}, taking its environment as input and outputting a phenotype, its kryptotype can be thought of as the mechanism inside the black box, as its specifics are hidden directly from selection. We will discuss the properties of kryptotypes, and primarily focus on the non-uniqueness and diversity of the kryptotype -- how the kryptotype can evolve and influence a myriad of evolutionary and biological phenomena such as speciation and evolvability.

  In order to model this mathematically, we represent a phenotype as the temporal expression patterns of a set of molecular species, most commonly proteins and mRNA. Can an organism express the enzymes and other relevent moleclar species for metabolism efficiently, given the presence of a metabolite? That is the phenotype is a given number of protein expressing genes and their regulatory interactions. In this model, we only focus on the evolution of regulation, and ignore mutations in protein function. For mathematical tractability these regulatory interactions are represented as a linear system with evolutionarily pliable regulatory coefficients.

  \subsection{An organism is defined by its molecular time-dynamics}
   Molecular (say protein or mRNA) concentrations are represented by $\vec{x}$, the gene regulatory interactions by $\alpha, \beta, \gamma,$ and $\delta,$ and the environmental/initial inputs by $B$.  
      \begin{equation*}
        \begin{bmatrix} \dot{x}_{1} \\ \dot{x}_{2} \end{bmatrix} = \begin{bmatrix} \alpha & \beta \\ \gamma & \delta \end{bmatrix} \begin{bmatrix} x_{1} \\ x_{2} \end{bmatrix} + \begin{bmatrix} B_{1} \\ B_{2} \end{bmatrix} u
      \end{equation*}
    Each row vector $\begin{bmatrix} \alpha & \beta \end{bmatrix}$ of the GRN matrix is the promoter region of a gene, such that, $\alpha$ describes the how much gene-1 self regulates and $\beta$ describes how much gene-2 regulates gene-1. 
  \subsection{The phenotype is just a subset of an organism's proteins directly visible to selection}
    \begin{equation*}
      y = \begin{bmatrix} C_{1} & C_{2} \end{bmatrix} \begin{bmatrix} x_{1} \\ x_{2} \end{bmatrix}
    \end{equation*}
    For instance, if $C = \begin{bmatrix} 1 & 0 \end{bmatrix}$, the phenotype is only described by the dynamics of gene-1. This means that selection can only ``see'' what happens to the proteins produced by gene-1, whereas the dynamics of gene-2 are invisible to selection (and thus not formally defined as part of the phenotype).
      
      This does not mean, however, that gene-2 is unnecessary with respect to the phenotype!

        \begin{example}
          For example, consider a GRN with oscillating protein concentrations.
    In this 2 gene system, neither gene-1 nor gene-2 self regulate. Gene-2 up-regulates gene-1, and gene-2 is down-regulated by gene-1:
      \begin{equation*}
        \begin{bmatrix} \dot{x}_{1} \\ \dot{x}_{2} \end{bmatrix} = \begin{bmatrix} 0 & 1 \\ -1 & 0 \end{bmatrix} \begin{bmatrix} x_{1} \\ x_{2} \end{bmatrix} + \begin{bmatrix} 1 \\ 1 \end{bmatrix} u
      \end{equation*}
      Furthermore, only the dynamics of gene-1 are important:
      \begin{equation*}
        y = \begin{bmatrix} 1 & 0 \end{bmatrix} \begin{bmatrix} x_{1} \\ x_{2} \end{bmatrix}
      \end{equation*}
    The dynamics of gene-1 are, $y = \cos(t) + \sin(t)$.

    A single-gene system cannot oscillate.
  \end{example}
    \subsection{Expression dynamic complexity is limited by the number of genes in a network.}
    $f(t)$ describes protein concentrations at time $t$, $\lambda_{j}$ is the $j$th eigenvalue of the GRN and $\xi_{j}$ is a variable.
      \begin{equation*}
        f(t) = \sum_{j=1}^{n} e^{\lambda_{j}t} \xi_{j}
      \end{equation*}
      Sometimes it's (visually) easier to study the Laplace transform of the system. 
      \begin{equation*}
        \mathcal{L}(f(t)) := H(z) = \sum_{j}^{n} \frac{c_{j} b_{j}}{(z- \lambda_{j})}
      \end{equation*}

  \subsection{Model details -- LE Systems}
      An LE System: Linear Evolving (TI) System (in evolutionary state $\tau$), 
    \begin{align*}
      \Sigma(\tau) = \left\{ \begin{array}{cc} \dot{x} =& A(\tau)x + Bu \\
      y =& Cx \end{array} \right.
    \end{align*}
    \begin{align*}
      \mathcal{L}(\Sigma(\tau)) := H(z) = C \left(zI - A(\tau) \right)^{-1} B
    \end{align*}

      \begin{definition}[Phenomenological equivalence of systems]
    Let $(x(t),y(t))$ and $(\bar x(t),\bar y(t))$ be the solutions to \eqref{eqn:lti_system}
    with coefficient matrices $(A,B,C)$ and $(\bar A,\bar B,\bar C)$ respectively,
    and both $x(0)$ and $\bar x(0)$ are zero. 
    The systems defined by $(A,B,C)$ and $(\bar A,\bar B,\bar C)$ are
    \textbf{phenomenologically equivalent} 
    if
    \begin{align*}
        y(t) = \bar y(t) \qquad \text{for all} \; t \ge 0.
    \end{align*}
    Equivalently, this occurs if and only if
    \begin{align*}
        H(z) = \bar H(z)  \qquad \text{for all} \; z \ge 0,
    \end{align*}
    where $H$ and $\bar H$ are the transfer functions of the two systems.
\end{definition}

 \subsection{Gene Network Diversity}
      First we need to describe the phase space. How many different gene networks can do exactly the same thing? Rather, can we describe the set of phenotypically invariant gene regulatory networks -- the set of all gene networks that produce identical outputs, when given identical inputs. In the linear case we can analytically describe all the networks in this space,
      \begin{align*}
        \mathcal{A} := \{ VAV^{-1}, CV, VB : V \in GL_{n}(\mathbb{R}) \}
      \end{align*}
      for the minimal dimension, and for the general (including non-minimal) case as,
      \begin{align*}
        CA^{k}B = \widehat{C} \widehat{A}^{k} \widehat{B} \qquad .
      \end{align*}
      where $A$ is a decription of gene network topology, and $B$ and $C$ transform the system intputs to and outputs from the network, respectively. 

  \subsection{Constant Selection Through Time}

    A large population, perfectly adapted to it's environment, under constant selection, will be phenotypically invariant for all generations (given the circumstances outlined above). Despite a conservation of phenotype, variation at the genetic and interemediate molecular levels is possible and expected through the course of evolution, as shown below. Under these constraints, phenotypic variance can only be the consequence of genetic drift (stochastic evolutionary forces overpowering selective ones due to population size) or adaptation to a novel phenotype for a different strategy (signatures of each will be discussed).

      For convenience, we hold $B$ (input transformation) and $C$ (output transformation) constant, considering only systems with varying internal mechanics.
      \begin{align*}
        P(\vec{r}) \in \{ GL_{n}(\mathbb{R}) : CP = C \text{and } BP = B \} \\
        A(\vec{r}) := P(\vec{r}) A P^{-1}(\vec{r})
      \end{align*}

      Where $A := A(0)$ be the initial and minimal configuration of a population's gene network.
      Where $\vec{r}$ is the unique vector of free variables associated with any instantiation of $P$.
      Elements of $\vec{r}$  can be restricted to a biologically reasonable range, $s \leq r_{k} \leq l$ to prevent unrealistic topologies.
      From this persepctive, the values of $A(r)$ can be written as the $n^{2}$ vector $\vec{a}(r) = \text{vec}(A(r))$. In this space, evolution acts as a random walk on $r$. 
      Furthermore, let $\Delta a(r) := \lVert \frac{d}{dr} a(r) \rVert^{-1}$. $\Delta a(r)$ describes the average magnitude of neutral genetic change required to reach a phenotypically invariant gene network. Systems should neutrally drift and rewire at rates inversely proportional to this value, as at high values of $\Delta$, more mutations are required to rewire neutrally -- that is the fitness valley is wider. Given enough time, we expect evolution to traverse the space for all of $r$, albeit at different rates and probabilities.

      Perhaps the rate of phenotypically invariant evolution near state $\tau$ can be expressed as,
      \begin{align*}
        \Delta_{r} :&= \left\lVert \frac{d}{d \tau} \text{vec}\left[A(\tau)\right] \right\rVert^{-1} \\
      \end{align*}
      Thus, $\mu \Delta_{\tau}$ is the rate at which the kryptotype evolves near $\tau$ with mutation rate $\mu$.
      \jss{Maybe need to distinguish between differentiating from the left and right of $\tau$ since we want to know the rate and direction of evolution. i.e. $\partial_{\pm} f(A(\tau))$}

      Therefore the system will evolve from $A(tau) \rightarrow A(\tau \pm \mu \Delta_{\tau})$ stochastically through generational time.

      \subsection{Gene Network Growth and Decay}

      Consider the minimal 2-dimensional system $A_{[2]}(\tau) := P(\tau) \begin{bmatrix} \lambda_{1} & \lambda_{2} - k \\ 0 & \lambda_{2} \end{bmatrix} P^{-1}(\tau)$. With probability $\mu_{g}$ (network growth), a new regulatory gene is added to the system. 
        \begin{align*}
          A_{[2]}(\tau) \xrightarrow{\mu_{g}} A_{[3]}(\vec{\tau}) \\
          A_{[3]}(0) = \begin{bmatrix} A_{[2]}(\tau) & \begin{array}{cc} 0 \\ 0 \end{array} \\ \begin{array}{cc} 1 & 1 \end{array} & \gamma \end{bmatrix}
        \end{align*}
        After some time has passed, the GRN will reorganize to, 
        \begin{align*}
          P_{[3]}(\vec{r})A_{[3]}(0)P_{[3]}^{-1}(\vec{r}) \\
          P_{[3]}B_{[3]} = B_{[3]} \\
          C_{[3]}P_{[3]} = C_{[3]} \\
          P_{[3]}(\vec{r}) := \begin{bmatrix} 1 & 0 & 0 \\ 1 & 1-r_{1} & r_{2} \\ r_{3} & -r_{3} & r_{4} \end{bmatrix} \\ 
        B_{[3]} := \begin{bmatrix} B_{[2]} \\ 0 \end{bmatrix} \\ 
        C_{[3]} := \begin{bmatrix} C_{[2]} & 0 \end{bmatrix}
        \end{align*}

        After some time, if the higher dimensional GRN is of the following form, and a submatrix of it, $\alpha \approx A(\tau)$, with probability $\mu_{d}$, the GRN can decay in size. 
        \begin{align*}
          A_{[3]}(\vec{r}) = \begin{bmatrix} \alpha_{[2]} & \begin{array}{cc} c_{1} \\ c_{2} \end{array} \\ \begin{array}{cc} c_{3} & c_{4} \end{array} & c_{5} \end{bmatrix}
        \end{align*}

        \begin{align*}
            v_{0} := \text{vec}(A_{2}(\tau)) &= \left(\begin{array}{c} \frac{\tau}{\tau - 1} - 1\\ \frac{\tau \left(3 \tau - 2\right)}{\tau - 1} - \tau \\ - \frac{1}{\tau - 1} \\ -\frac{3 \tau - 2}{\tau - 1} \end{array}\right) \\
          v_{1} := \text{vec}(\alpha_{[2]}(\vec{r})) &= \left(\begin{array}{c} -\frac{r_{4}}{r_{2} r_{3} - r_{4} \left(r_{1} - 1\right)}\\ \frac{\left(2 r_{1} - 1\right) \left(r_{1} + r_{2}\right)}{r_{1} - 1} + \frac{r_{2} r_{3} \left(\gamma - r_{2} + 2\right) - r_{1} r_{2} r_{3} \left(\gamma + 3\right)}{\left(r_{2} r_{3} - r_{4}\ \left(r - 1\right)\right)\ \left(r - 1\right)}\\ \frac{r_{4}}{r_{2} r_{3} - r_{4}\ \left(r_{1} - 1\right)}\\ \frac{\left(r_{4} + r_{2}\ r_{3}\right)\ \left(\gamma_{1} + 3\right) - r_{4}\ \left(\gamma - r_{2} + 2\right)}{r_{4} - r_{1}\ r_{4} + r_{2}\ r_{3}} - 3 \end{array}\right) 
        \end{align*}
        If, 
        \begin{align*}
          \lVert v_{0} - v_{1} \rVert < \delta \\
        \end{align*}
        Then, 
        \begin{align*}
          A_{[3]} \xrightarrow{\mu_{d}} A_{[2]}
        \end{align*}

        Or perhaps, what is the expected difference after time?
        \begin{align*}
          \mathbb{E}_{t} \left[ \lVert v_{0} - v_{1} \rVert \right] = x_{t}
        \end{align*}
        Does $x_{t}$ increase with time?

        \begin{align*}
          \text{Tr} \left[ \text{submatrix}_{33} \left(A_{[3]}\right) \right] = \frac{r_{2} \left(-(\lambda_{1} + \lambda_{2}) r_{3} + r_{4} + \gamma r_{3}\right)}{r_{4} - r_{1}r_{4} + r_{2}r_{3}} + (\lambda_{1} + \lambda_{2})
        \end{align*}

        The difference between the trace of the sub-matrix in $A_{[3]}$ and any matrix in $A_{[2]}$ is the term on the left. Since the trace determines if the sub-matrix has the same eigenvalues as a minimal system, if that number grows with time since duplication, then we will see a ratchet. 
\bibliographystyle{plainnat}
\bibliography{krefs}

\end{document}
