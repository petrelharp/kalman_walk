\documentclass{article}
\usepackage{amsmath, amssymb, color, xcolor, amsthm}
\usepackage{graphicx, wrapfig, float, caption, dsfont, bbm, xfrac}
\usepackage{fullpage}
\usepackage{natbib}

\newcommand{\jss}[1]{{\color{olive}\it #1}}
% \newcommand{\ddt}{\frac{d}{dt}}
\newcommand{\ddt}{\dot}
\newcommand{\ro}{{ro}}
\newcommand{\nro}{{\bar{r}o}}
\newcommand{\rno}{{r\bar{o}}}
\newcommand{\nrno}{{\bar{r}\bar{o}}}
\newcommand{\reachable}{\mathcal{R}}
\newcommand{\unobservable}{\bar{\mathcal{O}}}
\newcommand{\R}{\mathbb{R}}
\newcommand{\E}{\mathbb{E}}
\renewcommand{\P}{\mathbb{P}}
\newcommand{\var}{\mathop{\mbox{Var}}}
\newcommand{\cov}{\mathop{\mbox{Cov}}}
\newcommand{\tr}{\mathop{\mbox{tr}}} % trace
\newcommand{\pda}{\frac{\partial}{\partial A_{ij}}}
\newcommand{\ind}{\mathds{1}}
\newcommand{\grad}{\nabla}

\newcommand{\calH}{\mathcal{H}}
\newcommand{\diag}{\text{diag}}
\newcommand{\1}{\mathbbm{1}}

% fitness as a fn of distance
\newcommand{\fit}{\mathcal{F}}
% fitness as a fn of A
\newcommand{\fitx}{\mathcal{F}}
% set of optimal coefficients
\newcommand{\optx}{\mathcal{X}}
% optimal phenotype
\newcommand{\optph}{\Phi_0}
% distance in phenotype space
\newcommand{\dph}{d}
% incompatibility
\newcommand{\Incompat}{\mathcal{I}}

\DeclareMathOperator{\spn}{span}

\newtheorem{theorem}{Theorem}
\newtheorem{lemma}{Lemma}
\newtheorem{definition}{Definition}
\newtheorem{example}{Example}

\begin{document}

%%%%%%%%%%%%%%%%%%%%%%%%%%%%%%%
\section{Local expansion of the fitness surface}
\label{apx:H_calc}

Suppose that $\rho(t) \ge 0$ is a weighting function on $[0,\infty)$
so that fitness is a function of $L^2(\rho)$ distance of the impulse response from optimal.
With $h_0(t) = C_0 e^{tA_0} B_0$ a representative of the optimal set:
\begin{equation}
    \begin{aligned}
        D(A, B, C)^2
        &:= 
        \int_0^\infty \rho(t) \left| h_A(t) - h_0(t) \right|^2 dt \\
        &:= 
        \int_0^\infty \rho(t) \left| C e^{At} B - C_0 e^{A_0 t} B_0 \right|^2 dt \\
        &= 
        \int_0^\infty \rho(t) \tr\left\{
            \left( C e^{At} B - C_0 e^{A_0 t} B_0 \right)^T
            \left( C e^{At} B - C_0 e^{A_0 t} B_0 \right)
        \right\} dt \\
        &= 
        \int_0^\infty \rho(t) \tr\left\{
            \left( C e^{At} B - C_0 e^{A_0 t} B_0 \right)
            \left( C e^{At} B - C_0 e^{A_0 t} B_0 \right)^T
        \right\} dt  ,
        % &= 
        % \int_0^\infty \rho(t)  \tr\left\{
        %         C e^{At} B B^T e^{A^T t} C^T 
        %         - C_0 e^{A_0t} B_0 B^T e^{A^T t} C^T 
        %         \right. \\ &\qquad \qquad \left. {}
        %         - C e^{At} B B_0^T e^{A_0^T t} C_0^T 
        %         + C_0 e^{A_0t} B_0 B_0^T e^{A_0^T t} C_0^T 
        % \right\} dt \\
        % \int_0^\infty \rho(t) \left| C \left( e^{At} - e^{A_0 t} \right) B \right|^2 dt \\
        % &= 
        % \int_0^\infty \rho(t) C \left( e^{At} - e^{A_0 t} \right) B B^T \left( e^{At} - e^{A_0 t} \right)^T C^T dt
    \end{aligned}
\end{equation}
where $\tr X$ denotes the trace of a square matrix $X$.
How does this change as we perturb about $(A_0, B_0, C_0)$?
First we differentiate with respect to $A$, keeping $B=B_0$ and $C=C_0$ fixed.
Since
\begin{equation}
  \begin{aligned}
      \frac{d}{du} e^{(A+uZ)t} \vert_{u=0}
      &=
      \int_0^t e^{As} Z e^{A(t-s)} ds, 
  \end{aligned}
\end{equation}
we have that
\begin{equation}
  \begin{aligned}
      \frac{d}{du} D(A+uZ,B_0,C_0)^2 \vert_{u=0}
      &=
        2 \int_0^\infty \rho(t) \tr\left\{ C_0 \left( \int_0^t e^{As} Z e^{A(t-s)} ds \right) B_0 B_0^T \left( e^{At} - e^{A_0 t} \right)^T C_0^T \right\} dt \\
      &=
        2 \int_0^\infty \rho(t) \tr\left\{ C_0 \left( \int_0^t e^{As} Z e^{A(t-s)} ds \right) B_0 \left( h_A(t) - h_0(t) \right)^T \right\} dt 
  \end{aligned}
\end{equation}
and, by differentiating this and supposing that $A$ is on the optimal set,
i.e., $h_A(t)=h_0(t)$, (so without loss of generality, $A=A_0$):
\begin{equation}
  \begin{aligned}
      \calH^{A,A}(Y,Z) 
      &:= 
      \frac{1}{2} \frac{d}{du} \frac{d}{dv} D(A_0+uY+vZ,B_0,C_0)^2 \vert_{u=v=0} \\
      &=
        \int_0^\infty \rho(t) \tr\left\{ C_0 
        \left( \int_0^t e^{A_0 s} Y e^{A_0 (t-s)} ds \right) 
        B_0 B_0^T 
        \left( \int_0^t e^{A_0 s} Z e^{A_0 (t-s)} ds \right)^T
        C_0^T \right\} dt  .
  \end{aligned}
\end{equation}

The function $\calH$ will define a quadratic form.
To illustrate the use of this, suppose that $B$ and $C$ are fixed.
% Here $\calH$ is the quadratic form underlying the Hamiltonian.
By defining $\Delta_{ij}$ to be the matrix with a 1 in the $(i,j)$th slot
and 0 elsewhere,
the coefficients of the quadratic form is
\begin{equation}
    \begin{aligned}
        H_{ij, k\ell}(A)
        &:=
        \calH(\Delta_{ij}, \Delta_{k\ell}) .
    \end{aligned}
\end{equation}

We could use this to get the quadratic approximation to $D$ near the optimal set.
To do so, it'd be nice to have a way to compute the inner integral above.
Suppose that we diagonalize $A = U \Lambda U^{-1}$.
Then
\begin{equation} \label{eqn:exp_deriv}
  \begin{aligned}
      \int_0^t e^{As} Z e^{A(t-s)} ds 
      &=
      \int_0^t U e^{\Lambda s} U^{-1} Z U e^{\Lambda (t-s)} U^{-1} ds 
  \end{aligned}
\end{equation}
Now, notice that
\begin{equation}
  \begin{aligned}
      \int_0^t e^{s \lambda_i} e^{(t-s) \lambda_j} ds
      &=
      \frac{ e^{t \lambda_i} - e^{t \lambda_j} }{ \lambda_i - \lambda_j } 
          \qquad & \text{if} \quad i \neq j \\
      &=
          t e^{t \lambda_i} 
          \qquad & \text{if} \quad i = j 
  \end{aligned}
\end{equation}
Therefore, 
defining
\begin{equation}
    \begin{aligned}
    X_{ij}(t,Z) 
       &= 
        \left( U^{-1} Z U \right)_{ij}
      \frac{ e^{t \lambda_i} - e^{t \lambda_j} }{ \lambda_i - \lambda_j } 
          \qquad & \text{if} \quad i \neq j \\
      &=
          \left( U^{-1} Z U \right)_{ii}
          t e^{t \lambda_i} 
          \qquad & \text{if} \quad i = j 
    \end{aligned}
\end{equation}
moving the $U$ and $U^{-1}$ outside the integral and integrating we get that
\begin{equation}
  \begin{aligned}
      \int_0^t e^{As} Z e^{A(t-s)} ds 
      &=
      U X(t,Z) U^{-1} .
  \end{aligned}
\end{equation}
This implies that
\begin{equation}
    \begin{aligned}
        D(A_0+\epsilon Z)^2
        &\approx \frac{1}{2} \epsilon^2 
        \int_0^\infty
            \rho(t) \tr\left\{ C U X(t,Z) U^{-1} B B^T (U^{-1})^T X(t,Z)^T U^T C^T \right\}
        dt .
    \end{aligned}
\end{equation}

To compute the $n^2 \times n^2$ matrix $H$,
we see that if $Z=\Delta_{k \ell}$, then
\begin{equation}
  \begin{aligned}
      X_{ij}^{k\ell}(t) 
      &= 
      (U^{-1})_{\cdot k} U_{\ell \cdot}
      \frac{ e^{t \lambda_i} - e^{t \lambda_j} }{ \lambda_i - \lambda_j } 
          \qquad & \text{if} \quad i \neq j \\
      &=
      (U^{-1})_{\cdot k} U_{\ell \cdot}
      t e^{t \lambda_i} 
          \qquad & \text{if} \quad i = j 
  \end{aligned}
\end{equation}
where $U_{k \cdot}$ is the $k$th row of $U$,
and so
\begin{equation}
    \begin{aligned}
        H_{ij, k\ell}(A)
        &=
        \int_0^\infty
            \rho(t) \tr\left\{ C U X^{ij}(t) U^{-1} B B^T (U^{-1})^T X^{k\ell}(t)^T U^T C^T \right\}
        dt .
    \end{aligned}
\end{equation}
This implies that
\begin{equation}
    \begin{aligned}
        D(A_0+\epsilon Z)^2
        &\approx \frac{1}{2} \epsilon^2 \sum_{ijk\ell} H_{ij,k\ell}(A_0) Z_{ij} Z_{k\ell}  .
    \end{aligned}
\end{equation}

By section \ref{ss:quant_gen},
if we set $\Sigma=\sigma^2 I$ and $U=H$,
then a population at $A_0+Z$ experiences a restoring force of strength
$(I + \sigma^2 H^{-1})^{-1} Z$ (treating $Z$ as a vector and $H$ as an operator on these).
If $\sigma^2$ is small compared to $H^{-1}$
then this is approximately $-\sigma^2 H^{-1} Z$.
This suggests that the population mean follows an Ornstein-Uhlenbeck process,
as described (in different terms) in \citet{hansen1996translating}.

More generally, $B$ and $C$ may also change.
To extend this we need the remaining second derivatives of $D^2$.
First, in $B$:
\begin{equation}
    \begin{aligned}
        \mathcal{H}^{B,B}(Y,Z)
        &:= 
        \frac{1}{2} \frac{d}{du} \frac{d}{dv} D(A_0,B_0+uY+vZ,C_0)\vert_{u=v=0} \\
        &=
        \frac{1}{2} \int_0^\infty \rho(t) \tr\left\{ 
        C_0 e^{t A_0}
        \frac{d}{du} \frac{d}{dv} 
        (uY+vZ)
        (uY+vZ)^T
        \vert_{u=v=0} 
        e^{t A_0^T} C_0^T 
        \right\} dt  \\
        &=
        \frac{1}{2} \int_0^\infty \rho(t) \tr\left\{ 
        C_0 e^{t A_0}
        \left( Y Z^T + Z Y^T \right)
        e^{t A_0^T} C_0^T 
        \right\} dt  .
  \end{aligned}
\end{equation}
Next, in $C$:
\begin{equation}
    \begin{aligned}
        \mathcal{H}^{B,B}(Y,Z)
        &:= 
        \frac{1}{2} \frac{d}{du} \frac{d}{dv} D(A_0,B_0,C_0+uY+vZ)\vert_{u=v=0} \\
        &=
        \frac{1}{2} \int_0^\infty \rho(t) \tr\left\{ 
        B_0 e^{t A_0^T}
        \frac{d}{du} \frac{d}{dv} 
        (uY+vZ)^T
        (uY+vZ)
        \vert_{u=v=0} 
        e^{t A_0} B_0
        \right\} dt  \\
        &=
        \frac{1}{2} \int_0^\infty \rho(t) \tr\left\{ 
        B_0 e^{t A_0^T}
        \left( Y Z^T + Z Y^T \right)
        e^{t A_0} B_0
        \right\} dt  .
  \end{aligned}
\end{equation}
Now, the mixed derivatives in $B$ and $C$:
\begin{equation}
    \begin{aligned}
        \mathcal{H}^{B,C}(Y,Z)
        &:= 
        \frac{1}{2} \frac{d}{du} \frac{d}{dv} D(A_0,B_0+uY,C_0+vZ)\vert_{u=v=0} \\
        &=
        \int_0^\infty \rho(t) \tr\left\{ 
        Y e^{t A_0^T} C_0^T Z e^{t A_0} B_0
        \right\} dt  .
  \end{aligned}
\end{equation}
In $A$ and $B$
\begin{equation}
    \begin{aligned}
        \mathcal{H}^{A,B}(Y,Z)
        &:= 
        \frac{1}{2} \frac{d}{du} \frac{d}{dv} D(A_0+uY,B_0+vZ,C_0)\vert_{u=v=0} \\
        &=
        \int_0^\infty \rho(t) \tr\left\{ 
        C_0 \left(\int_0^t e^{s A_0} Y e^{(t-s) A_0} ds \right) B_0
        Z^T e^{t A_0} C_0
        \right\} dt  ,
  \end{aligned}
\end{equation}
and finally in $A$ and $C$:
\begin{equation}
    \begin{aligned}
        \mathcal{H}^{A,C}(Y,Z)
        &:= 
        \frac{1}{2} \frac{d}{du} \frac{d}{dv} D(A_0+uY,B_0,C_0+vZ)\vert_{u=v=0} \\
        &=
        \int_0^\infty \rho(t) \tr\left\{ 
        C_0 \left(\int_0^t e^{s A_0} Y e^{(t-s) A_0} ds \right) B_0
        B_0 e^{t A_0} Z
        \right\} dt  .
  \end{aligned}
\end{equation}

Together, numerical computation of these expressions,
along with estimates of genetic covariance within a population,
allow precise predictions of evolutionary dynamics of a particular system.
The approximation should be good as long as the second-order Taylor approximation holds.

\bibliographystyle{plainnat}
\bibliography{krefs}

\end{document}
