\documentclass[11 pt]{article}
\usepackage{amsmath, amssymb, color}
\usepackage{graphicx}
\usepackage{fullpage}
\usepackage[hidelinks]{hyperref}

\begin{document}

% Outline:
% 
% - What the problems are:
% 
%     * evolution of perfectly fit populations
% 
% - Motivations: Can you infer function from structure? And...
% 
%     * motifs
%     * speciation
%     * network size
%     * rube-goldberginess
%     * redundancy, robustness
%     * evolvability related to minimality?
% 
% - Give a concrete, simple example.
% 
% - Modeling framework (LTI systems)
%
%     * y = "phenotype"
%     * x = "internal state" or "kryptotype"?
%     * A = "regulatory matrix" or "genetic network"?
%     * Kalman decomposition
% 
% - Evolution of free parameters in the Kalman decomposition
%
%     * Brownian motion on free parameters
%     * how mutations map into this space
%     * dimension: duplication, deletion, recruitment of new things
% 
% - (Compute things)
% 
% - (Allow non-fit stuff too.)

\newcommand{\plr}[1]{{\color{blue}\it #1}}
% \newcommand{\ddt}{\frac{d}{dt}}
\newcommand{\ddt}{\dot}

\section{Background}

\plr{Review of relevant other stuff.}

A linear time invariant (LTI) dynamical system is a system of linear
differential equations that describes a physical process. 
LTI systems are used in electrical and control engineering to model a myriad of phenomena including
circuits. 
One can use this methodology, under a set of assumptions, to reverse engineer a
mechanism from impulse experiments (input/output data). The idea is that given
a black box, and it's experimental manipulation, can we describe the internal
mechanisms? 

\section{The problem}

Organisms often have to respond to their environments,
and the way this happens is that external input triggers a cascade of molecular signals
whose eventual result is a reaction,
whose approprateness for the situation determines fitness.
\plr{A similar situation occurs in development (...?).}
However, there are many ways to skin a cat:
especially given a number of possible molecular intermediaries,
there may be a large number of ways to construct such a molecular signalling system
that has precisely the \emph{same} input-output relationships,
and hence the same function.
Therefore, any mutation that changes one such system to another, equivalent version
will be neutral,
and may well become polymorphic in a population if there are no ill effects in heterozygotes.
Certainly, mutations that move a system away from optimal functioning will also appear,
but in very large populations will not rise to high frequency.
\plr{Deleterious things are quite important but somehow justify starting with the neutral ones.}

\plr{
    More stuff on why we care about this:  
Provide neutral model; can address questions.  
Main point of paper.
}

\paragraph{Example: Oscillator.}
Motivate the system:
\begin{align*}
    A &= \left[\begin{matrix} 
        0 & 1 \\ 
       -1 & 0 
    \end{matrix}\right]
\end{align*}
as a special case of the below.


\paragraph{Example: two-component systems.}
As a more general example,
suppose that a cell produces two molecular species, $S_1$ and $S_2$ 
whose production are both stimulated by the presence of an external ``input'' substance.
\plr{call them TF?}
Suppose that the two species self-regulate with strengths $\lambda_1$ and $\lambda_2$, respectively,
and the second species also regulates the first species with strength $\gamma$.
\plr{Rephrase in terms of binding to promoter sequences?}
However, only the time course of the concentration of
the first one of these species determines the fitness of the organism.
(As happens for instance in \plr{XXX}.)
Write $x(t) = (x_1(t),x_2(t))$ for the concentrations of the two species at time $t$,
and write $u(t)$ for the concentration at time $t$ of the input substance.
Then, if the rates at which each species are produced
are linear functions of the concentrations
\plr{(look up way this is usually said in the literature.)},
then the dynamics of the system are given by
\begin{align*}
    \ddt x_1(t) &= \lambda_1 x_1(t) + \gamma x_2(t) + u(t) \\
    \ddt x_2(t) &= \lambda_2 x_1(t) + u(t) ,
\end{align*}
where $\ddt x$ denotes the time derivative.
The initial conditions, $x_1(0)$ and $x_2(0)$, 
and the input $u(t)$ then determine the concentrations through time.
If we record the regulatory coefficients in the matrix:
\begin{align*}
    A &= \left[\begin{matrix} 
        \lambda_1 & \gamma \\ 
       0 & \lambda_2  
    \end{matrix}\right],
\end{align*}
and define the column vector $B = [1,1]^T$,
then in matrix notation the dynamics are
\begin{align*}
    \ddt x(t) &= A x(t) + B u(t) .
\end{align*}
If these are all transcription factors,
and we suppose their binding motifs are fixed,
then the $i^\text{th}$ row of $A$ is determined by the $i^\text{th}$ promoter sequence.

\plr{Note that $B$ and $C$ are fixed here.}

Since the system is linear, the state of the system at any time
is the superposition of its responses to all previous inputs.
This implies that 
if $G(t)$ is the transient response of the system to a unit impulse,
$t$ units of time later,
and $x(0) = 0$,
then the time course of the concentration of the thing we care about, $x_1(t)$, 
can be written as
\begin{align} \label{eqn:ex_convolution}
    x_1(t) &= \int_0^t G(t-s) u(s) ds .
\end{align}
However, it turns out that 
\plr{look for simple demontration this is true}
\begin{align*}
    P_{p} = \left[\begin{matrix} 
        1 & 0 \\ 
        p & 1-p 
    \end{matrix}\right],
\end{align*}
then for any $p \neq 1$, 
if we replace the regulatory matrix $A$ with
\begin{align*}
    A(p) = P^{-1} A P,
\end{align*}
then the response of $x_1(t)$ to any particular input $u(t)$ is identical to the original system.
(However, $x_2$ may well be different!)

For instance, setting $p=-1$,
the system with
\begin{align*}
    A(-1) = \left[\begin{matrix} 
        \lambda_1 + \gamma/2 & \gamma/2 \\ 
        \lambda_2-\lambda_1-\gamma/2 & \lambda_2 - \gamma/2
    \end{matrix}\right],
\end{align*}
looks very different, but gives the same input-output relationship.
Although these systems are equivalent,
hybrids between them may not be:
\plr{do example}.


\section{Regulatory networks as linear, time-invariant systems}

\plr{Make sure to have strongly worded disclaimer somewhere
that we know real networks aren't linear
but they are locally.
Plus, it works in engineering.}

% - General description of the LTI process in biological context
%     * y = "phenotype"
%     * x = "internal state" or "kryptotype"?
%     * A = "regulatory matrix" or "genetic network"?
% - Why transfer function determines input-output relationship
% - Kalman decomposition in unique form

We will now lay out the model in more general terms.
Suppose that the \emph{internal state} of the system
is parameterized by the concentrations of a collection of $n$ molecular species,
$S_1, \ldots, S_n$,
and the vector of concentrations at time $t$ we denote $x(t)=(x_1(t),\ldots,x_n(t))$.
There are also $m$ ``input'' species, whose concentrations are determined
exogenously to the system,
and are denoted $u(t) = (u_1(t),\ldots,u_m(t))$,
and $\ell$ ``output'' species, whose concentrations are denoted
$y(t) = (y_1(t),\ldots,y_\ell(t))$.
The output is merely a linear function of the internal state:
\begin{align*}
    y_i(t) = \sum_j C_{ij} x_i(t).
\end{align*}
Since $y$ is what natural selection acts on, we refer to it as the \emph{phenotype},
and sometimes in contrast refer to $x$ as the \emph{kryptotype},
as it is ``hidden'' from direct selection.
\plr{Maybe.}
The rate at which the $i^\text{th}$ species is produced
is a weighted sum of the concentrations of the other species
as well as the input:
\begin{align*}
    \ddt x_i(t) = \sum_j A_{ij} x_j(t) + \sum_k B_{ik} u_k(t) .
\end{align*}
In matrix notation, this is written more concisely as
\begin{align*}
    \ddt x(t) &= A x(t) + B u(t) \\
    y(t) &= C x(t) .
\end{align*}

\plr{Everything we say I think is assuming $x(0)=0)$.  
Figure this out. ``Zero-state equivalence.''}

Given an initial condition and an input, 
it is possible to write the solution $x(t)$
as a convolution with an input-response kernel as in the example above
(equation \eqref{eqn:ex_convolution}).
An alternative way of describing the input-output relationship,
more common in engineering,
is instead to find the \emph{transfer function}
\begin{align*}
    H(z) = C(z I - A)^{-1} B,
\end{align*}
which is more mathematically readable \plr{if you know what it means}.
This can be thought of as the system's response to input at ``frequency'' $z$.
The fact that the transfer function uniquely determines the system's
input-output relationship (i.e., the mapping $u \mapsto y$)
follows from the fact that if $\tilde y(z) = \int_0^\infty e^{-zt} y(t) dt$
is the Laplace transform of $x$,
and $\tilde u(z)$ is the Laplace transform of $u$,
then
\begin{align*}
    \tilde y(z) = H(z) \tilde u(z) .
\end{align*}
(This happens because the Laplace transform takes convolutions to products,
and $H$ is the Laplace transform of the kernel $G$.)

\plr{Should we remind what dimensions everything is?}

The state space representation is considered an internal description of the
system, whereas the transfer function is an external description. 
``Realizations'' are something.

State explicitly that same transfer function iff same input-output relationship.

One way to get equivalent things is change of basis (``algebraic equivlance'').
If multiple realizations of a system are algebraically equivalent, then they
must also have identical transfer functions $G(s) = \overline{G}(s)$. However,
the opposite relation is not necessarily true. Multiple systems can have
identical transfer functions and not have algebraic equivlaence. In fact, two
systems with identical transfer functions can even have different dimensions. 

But, that's not all: Kalman decomposition.  
Theorem that all equivalent systems are of this form; 
and representation is unique.

Talk about minimal dimension: point out that are degrees of freedom 
even in minimal dimension with B and C fixed.

Give an example.



\section{What biological question are we asking?}

How do gene regulatory networks (GRNs) change through evolutionary time?
Specifically, under constant selection and environmental pressures, how does a
maximally fit population's GRN change and what molecular and population
parameters are significant to the process? How frequent does gene network
rewiring occur in the absense of genetic drift or adaptation? Is rewiring
typically compensatory and ususally follow the fixation of a deleterious
allele, or can networks neutrally reorganize? How does the size, average
degree, function, peiotropy, etc of a network influence its likelihood to
drift? If networks can drift, how much topological heterogeneity can a
well-mixed population tolerate, if any? Does topoligical heterogeneity and/or
the size of the neutral genotype-set, if consequences of network organization,
constrain or entail exaptation and evolvability? Are some network organizations
(maybe higher dimensional realizations?) able to more rapidly evolve to
construct novel phenotypes and meet the demands of changes in selection
pressures?

\section{How can we mathematically model this question?}

In some respect an organism can be thought of as a black box that responds to
some set of inputs (it's environment and other initial conditions) and outputs
a phenotype. The phenotype is then evaluated by natural selection. The black
box metaphor holds, because much of the details of what happens to construct a
phenotype are unimportant as far as selection is concered, so long as the
phenotype reliably performs its function. If multiple internal mechanisms can
map the same set of inputs to the same set of outputs, and these mechanisms are
mutationally nearby (in genotype space), then evolution may drift from
mechanism to mechanism. 

In this simple model we can say that $B\vec{u}$ is a list of initial GRN
protein concentrations determined by the more general initial environmental
conditions $\vec{u}$. $A$ represents all the genetic interactions of a
particular GRN, and $C\vec{x}$ is an organism's phenotype. Fitness could be
calculated by comparing a subset of the phenotype to an optimum. Maybe,
$f(\cdot) = e^{- \int_{a}^{b} \Vert G(s) - G^{*}(s) \Vert ds}$ Alternatively,
fitness scores could simply assess whether or not a phenotype is within some
range or breaks some threshold by some time point. Or possbily the simplest:
any phenotype not exactly equal to $G(s)$ is lethal, and $G(s)$ is perfectly
fit. 

Next, we will have to define mutation and recombination. Mutation needs to add
and remove genetic interactions and recombination needs to shuffle genes during
sexual reproduction. We could adapt methods from Lynch 2007 and Lynch and
Hagner 2015 to model the probability of a TFBS appearing or disappearing due to
mutational pressure. Every time a new binding site is gained or lost, the
values within the A matrix are either modified or rows and columns are added
removed following a specific set of rules. I am not sure if it would be easier
to maintain a matrix size (say $m \times m$) where $m$ is arbitrarily large,
with  most entries being zero, or if a matrix should only have as many
dimensions as the number of active TFs. Recombination should then shuffle rows
of the $A$ matrix randomly, as each row represents the regulatory region of a
given gene (assuming only cis-regulation, and that the regulatory element is
small enough to not break up during recombination).  

To study some of the above posed questions, we can simulate (or analytically
determine) the rates of GRN change, and how frequently these changes lead to
significant rewiring or speciation. We would want to know the dynamics of a
brownian motion over the set of all realizations.  Another interesting question
would be to determine the size of the neutral genotype space as well as the
number of connections from the neutral genotype space to the non-neutral
genotype space. Maybe higher dimensional realizations of a GRN will have
significantly more connections to non-neutral genotypes and therefore be more
``evolvable'' or exapted to novel environments? 

Let $M \in \mathbb{R}^{n \times n}$ be a least dimensional realization of $A$,
and let $M'$ bean $(n+m \times n+m)$ matrix where $M'_{(i,j)} = M_{(i,j)} \
\text{for } i,j \leq n, \ \text{and } 0 \ \text{otherwise}$. Let $P$ be any
$(n+m \times n+m)$ matrix with rank $n$. Then for any matrix $B$ such that $BP
= PM'$, $B$ is a realization of $A$.     

\section{List of papers with some notes.}

R. E. Kalman, “Mathematical description of linear dynamical systems,” Journal of the Society for Industrial and Applied Mathematics, Series A: Control, vol. 1, no. 2, pp. 152–192, 1963.
This is a good overview written by Kalman. Probably the most important paper on this list.

Minimal state-space realization in linear system theory: an overview. B. De Schutter. 2000.
Another, more recent, overview that I found helpful to understand controllability and observability. 

The Inverse Problem of Stationary Covariance Generation. BDO Anderson. 1969.

Equivalence of Linear Time- Invariant Dynamical Systems. BDO Anderson, BW Necomb, RE Kalman. 1966. 

B. Ho and R. E. Kalman, “Effective construction of linear state-variable models from input/output functions,” Proceedings of the 3rd Annual Allerton Conference on Circuit and System Theory, vol. 1, no. 2, p. 449459, 1965.

Dynamic Structure Functions for the Reverse Engineering of LTI Networks. J Goncalves. 2007.
I have not read this paper yet, but apparently it outlines a method for recovering the state space equation from specific sets of measurements.

New Developments in Systems Theory Relevant to Biology. RE Kalman. 1968.
Not technical but kind of interesting. 

Old and New Directions of Research in Systems Theory. RE Kalman. 2010?

Minimality and Observability of Group Systems. HA Loeliger, GD Forney Jr., T Mittelholzer, MD Trott. 1994. 
Another paper I just found. It looks like it might be useful.

Partial Realization of Descriptor Systems. P Benner and VI Sokolov. 2006. 
I think this is about recovering a finite realization from infinite systems. Maybe it would be interesting to think about the similarity between transfer functions though? If selection only cares about a specific interval of the transfer function, say $G(10)$, then maybe two partial transfer functions will be effecrively equivlaent? 

There are also the previous papers I sent on identifiability and distinguishability. In the original Bellman and Astrom paper where they introduce ``structural identifiability'' they show that a linear dynamical system is identifiabile if and only if it is also controllable and observable. 




\end{document}
